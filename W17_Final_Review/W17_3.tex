\documentclass[t]{beamer}
\usepackage[utf8]{inputenc}
\usepackage{xeCJK}
\usepackage{graphicx}
\usepackage{subfigure}
\usepackage{mathtools}
\usepackage{utopia} %font utopia imported
\usetheme{CambridgeUS}
\usecolortheme{dolphin}
\usefonttheme{professionalfonts}
\usepackage{natbib}
\usepackage{hyperref}
\usepackage{fontspec}
\usepackage{setspace}
\usepackage{float}
\usepackage{extarrows}
\usepackage{tabularx}
% \usepackage{enumitem}

\setCJKmainfont{SourceHanSansSC-Regular.otf}[Path=../fonts/, BoldFont=bold.otf]

\setbeamerfont{title}{size=\Large}
\setbeamerfont{subtitle}{size=\small}
\setbeamerfont{date}{size=\small}
\setbeamerfont{institute}{size=\small}

\setstretch{1.3}
% \setlength{\parindent}{2em}
% \setlength{\parskip}{0pt}
% \setlist[itemize]{leftmargin=2em}

% ↓↓↓ Modify this ↓↓↓
\title{高等数学I 习题课17}
\subtitle{2024秋 期末卷讲评}
\date[2026.1.6]{2025.1.6}
% ↑↑↑ Modify this ↑↑↑

\author[上海科技大学]{}
\institute[]{上海科技大学}

\begin{document}

\begin{frame}[c]
    \vspace{15pt}
    \titlepage
\end{frame}
\section{单项选择题}

\begin{frame}
    1. 设 $f(x)$ 是定义在 $[a,b]$ 上的函数,下列说法正确的是(\quad)
    
    \small A. 若 $f(x)$ 连续,则一定存在函数 $F(x)$ 使得 $F'(x)=f(x)$
    
    B. 即便 $f(x)$ 存在可去间断点,依然可能存在函数 $F(x)$ 使得 $F'(x)=f(x)$
    
    C. 即便 $f(x)$ 存在跳跃间断点,依然可能存在函数 $F(x)$ 使得 $F'(x)=f(x)$
    
    D. 即便 $f(x)$ 存在第二类间断点,也一定存在函数 $F(x)$ 使得 $F'(x)=f(x)$
\end{frame}

\begin{frame}
    2. 下列说法错误的是(\quad)
    
    A. 若函数 $f(x)$ 在区间 $[a,b]$ 上连续,则 $f(x)$ 在该区间上可积
    
    B. 若函数 $f(x)$ 在区间 $[a,b]$ 上单调,则 $f(x)$ 在该区间上可积
    
    C. 若函数 $f(x)$ 在区间 $[a,b]$ 上可积,且令 $F(x)=\int_{a}^{x}f(t)dt$,则 $F'(x)=f(x)$
    
    D. 若函数 $F(x)$ 在区间 $(a,b)$ 有连续导数,令 $c\in(a,b)$ 为一个常数,则 $F(x)-\int_{c}^{x}F'(t)dt$ 有任意阶导数
\end{frame}

\begin{frame}
    3. 下列反常积分中收敛的是(\quad)
    \vspace{0.3cm}

    \begin{tabularx}{\textwidth}{XX}
        A. $\displaystyle \int_{-\infty}^{\infty}xe^{-x}dx$ & 
        B. $\displaystyle \int_{-\infty}^{\infty}xe^{-x^{2}}dx$ \\[1.5em]
        C. $\displaystyle \int_{-1}^{1}xe^{\frac{1}{x}}dx$ & 
        D. $\displaystyle \int_{-1}^{1}\frac{1}{x}e^{x}dx$
    \end{tabularx}
\end{frame}

\begin{frame}
    
\end{frame}

\begin{frame}
    4. 设 $\{a_{n}\},\{b_{n}\}$ 为两个数列,且 $\displaystyle \lim_{n\rightarrow\infty}a_{n}=0$。则(\quad)

    
    A. 当 $\displaystyle\sum_{n=0}^{\infty}b_{n}$ 收敛时,$\displaystyle\sum_{n=0}^{\infty}a_{n}b_{n}$ 收敛
    
    B. 当 $\displaystyle\sum_{n=0}^{\infty}|b_{n}|$ 收敛时,$\displaystyle\sum_{n=0}^{\infty}a_{n}b_{n}$ 收敛
    
    C. 当 $\displaystyle\sum_{n=0}^{\infty}b_{n}^{2}$ 收敛时,$\displaystyle\sum_{n=0}^{\infty}a_{n}b_{n}$ 收敛
    
    D. 当 $\displaystyle\sum_{n=0}^{\infty}b_{n}$ 发散时,$\displaystyle\sum_{n=0}^{\infty}a_{n}b_{n}$ 发散
\end{frame}

\begin{frame}
    
\end{frame}

\begin{frame}
    5. 假设 $f(x)$ 是定义在 $\mathbb{R}$ 上的奇连续函数,而 $g(x)$ 是定义在 $\mathbb{R}$ 上的偶连续函数。考虑下列函数:
    \[ F(x)=\int_{-x}^{x}f(t)dt, \qquad G(x)=\int_{-x}^{x}g(t)dt \]
    则下列说法错误的是(\quad)
    \vspace{0.3cm}
    
    \begin{tabularx}{\textwidth}{XX}
       A. $F(x)$ 是偶函数 & B. $F(x)$ 是奇函数 \\
       C. $G(x)$ 是偶函数 & D. $G(x)$ 是奇函数
    \end{tabularx}
\end{frame}

\begin{frame}
    
\end{frame}

%----------------------------------------------------------------------------------------
%	SECTION 2: Fill in the Blanks
%----------------------------------------------------------------------------------------
\section{填空题}

\begin{frame}
    6. 求极限 
    \[ \lim_{n\rightarrow\infty}\left(\frac{1}{n^{2}+1}+\frac{2}{n^{2}+2^{2}}+\cdots+\frac{n}{n^{2}+n^{2}}\right) = \underline{\makebox[2cm]{}} \]
\end{frame}

\begin{frame}
    7. 设 $f(x)=1+x+2x^{2}+3x^{3}$,则 $f(x)$ 在 $x_{0}=1$ 处的 Taylor 级数是 \underline{\makebox[3cm]{}}。
    \vspace{1cm}
    
    8. 函数项级数 $\displaystyle \sum_{n=1}^{\infty}\frac{x^{2n}}{(n^{2}+1)e^{n}}$ 的收敛域是 \underline{\makebox[2cm]{}}。
\end{frame}

\begin{frame}
    9. 令 $a, b\in\mathbb{R}$,计算定积分 
    \[ \int_{0}^{\pi} \left[ a\cos^{2}x+b(\sin x+\cos x)^{2} \right] dx = \underline{\makebox[2cm]{}} \]
    \vspace{1cm}
\end{frame}

\begin{frame}
    10. 设 $f(t)$ 是定义在 $\mathbb{R}$ 上以 $\tau>0$ 为周期的连续函数。令
    \[ F(x)=e^{x}\int_{x^{2}}^{x^{2}+\tau}f(t)dt \]
    则 $F'(x)-F(x) = \underline{\makebox[2cm]{}}$。
\end{frame}

%----------------------------------------------------------------------------------------
%	SECTION 3: Calculation Problems
%----------------------------------------------------------------------------------------
\section{计算题}

\begin{frame}
    11. 计算极限:
    \[ \lim_{x\rightarrow0}\frac{\int_{0}^{x^{2}}(e^{t^{2}}-1)dt}{x[\ln(1+x)]^{2}\sin(x^{2})} \]
\end{frame}

\begin{frame}
    12. 计算不定积分:
    \[ \int\frac{x^{3}+5x^{2}+x+5}{(x^{2}+1)(x^{2}-4)}dx \]
\end{frame}

\begin{frame}
    13. 计算不定积分 $(x>0)$:
    \[ \int e^{2x}\arctan\sqrt{e^x-1}dx \]
\end{frame}

\begin{frame}
    14. 设 $D$ 为由曲线 $y=-(x-1)^{2}+1$ 和 $y=x^{2}$ 围成的区域。求 $D$ 绕 $x$ 轴一周所得旋转体的体积。
\end{frame}

%----------------------------------------------------------------------------------------
%	SECTION 4: Proof Problems
%----------------------------------------------------------------------------------------
\section{证明题}

\begin{frame}
    15. 假设 $\{a_{n}\},\{b_{n}\}$ 是各项均为正的数列。并且满足:
    \begin{enumerate}
        \item[(i)] $\displaystyle \lim_{n\rightarrow\infty}\frac{b_{n}}{n}=0$
        \item[(ii)] $\displaystyle \lim_{n\rightarrow\infty}b_{n}\left(\frac{a_{n}}{a_{n+1}}-1\right)=\lambda>0$
    \end{enumerate}
    基于条件 (i)(ii),完成以下证明:
    \begin{itemize}
        \item[(a)] 证明 $\displaystyle \lim_{n\rightarrow\infty}n\left(\frac{a_{n}}{a_{n+1}}-1\right)=+\infty$
        \item[(b)] 证明存在一个正整数 $N$,使得当 $n\ge N$ 时有 $\displaystyle \frac{a_{n+1}}{a_{n}}\le\left(\frac{n}{n+1}\right)^{2}$
        
        \vspace{0.2cm}
        \small(提示: 用反证法。假设存在一个无穷子列 $\{a_{n_{i}}\}$ 使得 $\displaystyle \frac{a_{n_{i}+1}}{a_{n_{i}}}>\left(\frac{n_{i}}{n_{i}+1}\right)^{2}$,试估计 $\displaystyle \lim_{i\rightarrow\infty}n_{i}\left(\frac{a_{n_{i}}}{a_{n_{i}+1}}-1\right)$)\normalsize
        \vspace{0.2cm}
        
        \item[(c)] 证明 $\displaystyle \sum_{n=1}^{\infty}a_{n}$ 收敛。
    \end{itemize}
\end{frame}

\begin{frame}
    
\end{frame}

\begin{frame}
    
\end{frame}

\begin{frame}
    
\end{frame}


\begin{frame}
    16. 设函数 $f$ 在 $[0,+\infty)$ 上连续,且对于任意 $0<a<b$ 满足
    \[ f\left(\frac{a+b}{2}\right)\le\frac{f(a)+f(b)}{2} \]
    记 $\displaystyle F(x)=\frac{1}{x}\int_{0}^{x}f(t)dt$。求证:
    \[ F\left(\frac{a+b}{2}\right)\le\frac{F(a)+F(b)}{2} \]
\end{frame}

\begin{frame}
    
\end{frame}

\begin{frame}
    17. 令 $1 < a < b$,且令 $f(x)$ 为区间 $[a,b]$ 上的连续函数。证明:存在 $\xi\in(a,b)$ 满足
    \[ \xi^{2}f(\xi)=\frac{ab}{b-a}\int_{a}^{b}f(x)dx \]
    (提示: 可尝试柯西中值定理)
\end{frame}

\begin{frame}
    
\end{frame}

\begin{frame}
    
\end{frame}


% -------------------------------------------------------
% \section*{}
% \begin{frame}
% \vspace{25pt}
% \[
% \text{\Huge }
% \]
% \end{frame}

\end{document}