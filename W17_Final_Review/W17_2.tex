\documentclass[t]{beamer}
\usepackage[utf8]{inputenc}
\usepackage{xeCJK}
\usepackage{graphicx}
\usepackage{subfigure}
\usepackage{mathtools}
\usepackage{utopia} %font utopia imported
\usetheme{CambridgeUS}
\usecolortheme{dolphin}
\usefonttheme{professionalfonts}
\usepackage{natbib}
\usepackage{hyperref}
\usepackage{fontspec}
\usepackage{setspace}
\usepackage{float}
\usepackage{extarrows}
\usepackage{tabularx}
% \usepackage{enumitem}

\setCJKmainfont{SourceHanSansSC-Regular.otf}[Path=../fonts/, BoldFont=bold.otf]

\setbeamerfont{title}{size=\Large}
\setbeamerfont{subtitle}{size=\small}
\setbeamerfont{date}{size=\small}
\setbeamerfont{institute}{size=\small}

\setstretch{1.3}
% \setlength{\parindent}{2em}
% \setlength{\parskip}{0pt}
% \setlist[itemize]{leftmargin=2em}

% ↓↓↓ Modify this ↓↓↓
\title{高等数学I 习题课17}
\subtitle{2023秋 期末卷讲评}
\date[2026.1.6]{2025.1.6}
% ↑↑↑ Modify this ↑↑↑

\author[上海科技大学]{}
\institute[]{上海科技大学}

\begin{document}

\begin{frame}[c]
    \vspace{15pt}
    \titlepage
\end{frame}

\section{单项选择题}

\begin{frame}
    1.函数 $\displaystyle f(x) = \frac{|x|}{1+e^{-x}}$ 有($\qquad$)
    
    % \begin{array}{ll}
    %     A. 一条斜渐近线 & B. 两条斜渐近线 \\
    %     C. 一条水平渐近线和一条斜渐近线 & D. 两条水平渐近线
    % \end{array}
    \[
    \begin{array}{llc}
        \text{A. 一条斜渐近线} & \text{B. 两条斜渐近线}\\
        \text{C. 一条水平渐近线和一条斜渐近线} & \text{D. 两条水平渐近线}
    \end{array}
    \]
\end{frame}

\begin{frame}
    2.以下反常积分中收敛的是($\qquad$)\vspace{5pt}
    
    \begin{tabularx}{0.85\textwidth}{XX}
        A. $\displaystyle \int_{0}^{e}\frac{\mathrm dx}{x \ln x \ln(\ln x)}$ & B. $\displaystyle \int_{0}^{e}\frac{\mathrm dx}{x(\ln x)^{3}}$ \\
        \vspace{1pt}C. $\displaystyle \int_{e}^{+\infty}\frac{\mathrm dx}{x \ln x \ln(\ln x)}$ &\vspace{1pt} D. $\displaystyle \int_{e}^{+\infty}\frac{\mathrm dx}{x(\ln x)^{3}}$
    \end{tabularx}
\end{frame}

\begin{frame}
    3.考察两个函数
    \[
    f(x)=\begin{cases}e^{-1/x^{2}},&x\ne0;\\ 0,&x=0,\end{cases}
    \]
    和 $g(x)=x f(x)$.则 $x = 0$ ($\qquad$)
    
    \begin{tabularx}{0.85\textwidth}{X}
        A. 是 $g(x)$ 的极值点 \\ B. 不是 $g(x)$ 的极值点,但是 $f(x)$ 的极值点 \\
        C. 无法确定是否 $g(x)$ 的极值点 \\ D. 不是 $g(x)$ 的极值点,也不是 $f(x)$ 的极值点
    \end{tabularx}
\end{frame}

\begin{frame}
    
\end{frame}

\begin{frame}
    4.定义函数 $\displaystyle f(x)=\int_{0}^{x}\sin t^{2} dt$.则($\qquad$)
    
    \begin{tabularx}{0.9\textwidth}{XX}
        A. $f(\sqrt{2\pi})<0, f(-\sqrt{2\pi})<0$ & B. $f(\sqrt{2\pi})<0, f(-\sqrt{2\pi})>0$ \\
        C. $f(\sqrt{2\pi})>0, f(-\sqrt{2\pi})<0$ & D. $f(\sqrt{2\pi})>0, f(-\sqrt{2\pi})>0$
    \end{tabularx}
\end{frame}

\begin{frame}
    5.若 $f(x)$ 在区间 $[a,b]$ 上可导.考虑以下四个关于其导函数 $f^{\prime}(x)$ 的判断:
    \begin{itemize}
        \item[(i)] $f^{\prime}(x)$ 有原函数;
        \item[(ii)] $f^{\prime}(x)$ 可积;
        \item[(iii)] $f^{\prime}(x)$ 连续;
        \item[(iv)] $f^{\prime}(x)$ 没有第一类间断点.
    \end{itemize}
    其中一定正确的有($\qquad$)个.
    
    \begin{tabularx}{0.9\textwidth}{XXXX}
        A. 1 & B. 2 & C. 3 & D. 4
    \end{tabularx}
\end{frame}

\begin{frame}
    
\end{frame}

\section{填空题}

\begin{frame}
    6.函数 $y=\cos^{2}x+\sec x$ 在区间 $[-\pi/3, \pi/3]$ 的最大值为 $\underline{\qquad\qquad\qquad}$.
\end{frame}

\begin{frame}
    7.函数 $\displaystyle y=\frac{1}{1+x^{2}}$ 的上凸区间为 $\underline{\qquad\qquad\qquad\qquad\qquad\qquad\quad}$.
\end{frame}

\begin{frame}
    8.求极限 $\displaystyle \lim_{n\rightarrow\infty}\left(\frac{n}{n^{2}+1^{2}}+\frac{n}{n^{2}+2^{2}}+\cdot\cdot\cdot+\frac{n}{n^{2}+n^{2}}\right) = \underline{\qquad\qquad\qquad}$.
\end{frame}

\begin{frame}
    9.微分方程
\end{frame}

\begin{frame}
    10.曲线 $y=e^{-ax^{2}}$ 与坐标轴所围成的第一象限的图形绕 $y$ 轴旋转一周所得到的旋转体体积为 1.则 $a = \underline{\qquad\qquad\qquad\qquad\qquad\qquad\qquad}$.
\end{frame}

\section{计算题}

\begin{frame}
    11.求极限
    \[
    \lim_{x\rightarrow+0}\frac{\int_{0}^{\sin x}\sqrt{\tan t}\mathrm dt}{\int_{0}^{\tan x}\sqrt{\sin t}\mathrm dt}.
    \]
\end{frame}

\begin{frame}
    12.计算定积分
    \[
    \int_{0}^{\pi}\frac{\mathrm dx}{\sin^{2}x+4\cos^{2}x}.
    \]
\end{frame}

\begin{frame}
    
\end{frame}

\begin{frame}
    13.计算不定积分
    \[
    \int\frac{x \mathrm dx}{(x+1)(x+2)(x+3)}.
    \]
\end{frame}

\begin{frame}
    
\end{frame}

\section{几何题}

\begin{frame}
    
    \begin{columns}
        \begin{column}{0.5\textwidth}
            15.参数方程
            \[
            \begin{cases}
            x(t)=2t-t^{2} \\
            y(t)=2t^{2}-t^{3} 
            \end{cases}
            \quad (0\le t\le2)
            \]
            围成一条封闭曲线(见右图).

            (1) 求曲线最右侧点的坐标;
    
            (2) 求曲线所围区域的面积.
        \end{column}
        \begin{column}{0.5\textwidth}
            \begin{figure}
                \centering
                \includegraphics[width=0.8\linewidth]{23_t15.png}
            \end{figure}
        \end{column}
    \end{columns}
    
    
    
\end{frame}

\begin{frame}

\end{frame}

\section{证明题}

\begin{frame}
    17.设函数 $f(x)$ 在闭区间 $[a,b]$ 上连续可导且 $f(a)=0$.设 $M$ 为 $|f(x)|$ 在区间 $[a,b]$ 上的最大值.
    
    证明不等式:
    \[
    M^{2}\le(b-a)\int_{a}^{b}(f^{\prime}(x))^{2}\mathrm dx.
    \]
\end{frame}

\begin{frame}
    
\end{frame}

\section{综合题}

\begin{frame}
    18.考察函数
    \[
    f(x)=\int_{0}^{x}\frac{\sin t}{t}\mathrm dt.
    \]
    这个函数叫作「正弦积分函数」.
    \begin{itemize}
        \item [(1)] 证明 $x_{k}=k\pi, k\ne0$ 是 $f(x)$ 的所有极值点,并指出其中哪些是极大值点,哪些是极小值点.
    \end{itemize}
\end{frame}

\begin{frame}
    18.考察函数
    \[
    f(x)=\int_{0}^{x}\frac{\sin t}{t}\mathrm dt.
    \]
    这个函数叫作「正弦积分函数」.
    \begin{itemize}
        \item [(2)] 设 $\{x_{k_{i}}\}$ 为 $x$ 正半轴的极大值点从小到大排列构成的数列,证明相应的极大值 $f(x_{k_{i}})$ 是单调递减数列.类似的,将 $x$ 正半轴的极小值点从小到大排列,证明相应的极小值构成单调递增数列.
    \end{itemize}
\end{frame}

\begin{frame}
    18.考察函数
    \[
    f(x)=\int_{0}^{x}\frac{\sin t}{t}\mathrm dt.
    \]
    这个函数叫作「正弦积分函数」.
    \begin{itemize}
        \item [(3)] 写出 $f(x)$ 的拐点所要满足的等式,并证明任意两个相邻的极值点之间都有且仅有一个拐点(不用写出拐点的具体数值!).
    \end{itemize}
\end{frame}

\begin{frame}
    18.考察函数
    \[
    f(x)=\int_{0}^{x}\frac{\sin t}{t}\mathrm dt.
    \]
    这个函数叫作「正弦积分函数」.
    \begin{itemize}
        \item [(4)] 我们不加证明的承认以下等式
    \[
    \int_{0}^{\infty}\frac{\sin x}{x}\mathrm dx=\int_{0}^{\infty}\frac{1}{y^{2}+1}\mathrm dy.
    \]
    利用此等式写出 $f(x)$ 的渐近线公式.
    \end{itemize}
\end{frame}

\begin{frame}
    18.考察函数
    \[
    f(x)=\int_{0}^{x}\frac{\sin t}{t}\mathrm dt.
    \]
    这个函数叫作「正弦积分函数」.
    \begin{itemize}
        \item [(5)] 根据之前的答案,大致作图画出 $f(x)$ 的曲线.
    \end{itemize}
\end{frame}

% -------------------------------------------------------
% \section*{}
% \begin{frame}
% \vspace{25pt}
% \[
% \text{\Huge }
% \]
% \end{frame}

\end{document}