\documentclass[t]{beamer}
\usepackage[utf8]{inputenc}
\usepackage{xeCJK}
\usepackage{graphicx}
\usepackage{subfigure}
\usepackage{mathtools}
\usepackage{utopia} %font utopia imported
\usetheme{CambridgeUS}
\usecolortheme{dolphin}
\usefonttheme{professionalfonts}
\usepackage{natbib}
\usepackage{hyperref}
\usepackage{fontspec}
\usepackage{setspace}
\usepackage{float}
\usepackage{extarrows}
\usepackage{tabularx}
% \usepackage{enumitem}

\setCJKmainfont{SourceHanSansSC-Regular.otf}[Path=../fonts/, BoldFont=bold.otf]

\setbeamerfont{title}{size=\Large}
\setbeamerfont{subtitle}{size=\small}
\setbeamerfont{date}{size=\small}
\setbeamerfont{institute}{size=\small}

\setstretch{1.3}
% \setlength{\parindent}{2em}
% \setlength{\parskip}{0pt}
% \setlist[itemize]{leftmargin=2em}

% ↓↓↓ Modify this ↓↓↓
\title{高等数学I 习题课17}
\subtitle{2022秋 期末卷讲评}
\date[2026.1.6]{2025.1.6}
% ↑↑↑ Modify this ↑↑↑

\author[上海科技大学]{}
\institute[]{上海科技大学}

\begin{document}

\begin{frame}[c]
    \vspace{15pt}
    \titlepage
\end{frame}

\section{单项选择题}

\begin{frame}
    1.设函数 $f(x)$ 是 $\sin x$ 的一个原函数,则 $f(x)$ 的一个原函数是($\qquad$)
    
    \begin{tabularx}{0.9\textwidth}{XXXX}
        A. $-\cos x + 1$ & B. $\cos x + 1$ & C. $\sin x + x$ & D. $-\sin x + x$
    \end{tabularx}
\end{frame}

\begin{frame}
    2.函数 $f(x)=(x^{2}-3)\cdot e^{-x}$ 的单调增加区间为($\qquad$)
    
    \begin{tabularx}{0.9\textwidth}{XXXX}
        A. $(-\sqrt{3},\sqrt{3})$ & B. $(-\infty,-1)$ & C. $(-1,3)$ & D. $(3,+\infty)$
    \end{tabularx}
\end{frame}

\begin{frame}
    3.曲线 $y=\frac{1}{x}+\ln(1+e^{x})$ 有几条渐近线?($\qquad$)
    
    \begin{tabularx}{0.85\textwidth}{XXXX}
        A. 3 & B. 2 & C. 1 & D. 4
    \end{tabularx}
\end{frame}

\begin{frame}
    4.下列反常积分中收敛的是($\qquad$)\vspace{5pt}
    
    \begin{tabularx}{0.85\textwidth}{XX}
        A. \vspace{5pt}$\displaystyle\int_{1}^{+\infty} \frac{1}{\sqrt{x^2+x}} \mathrm dx$ & B. $\displaystyle\int_{0}^{1} \frac{1}{\sqrt{x^2+x}} \mathrm dx$ \\ 
        C. $\displaystyle\int_{-\infty}^{+\infty} \frac{x}{x^2+1} \mathrm dx$ & D. $\displaystyle\int_{-1}^{1} \frac1x \mathrm dx$ 
    \end{tabularx}
\end{frame}

\begin{frame}
    
\end{frame}

\begin{frame}
    5.设函数 $f(x)$ 为连续函数,对于两个命题:
    \begin{itemize}
        \item[(I)] 若 $F(x)=\int_{0}^{x}(\int_{0}^{u}[f(t)-f(-t)]\mathrm dt)\mathrm du$,则 $F(x)$ 为奇函数;
        \item[(II)] 若 $f(x)$ 为奇函数,则 $G(x)=\int_{0}^{x}[\int_{x}^{y}f(t^{3})\mathrm dt]\mathrm dy$ 为奇函数。
    \end{itemize}
    下列选项正确的是($\qquad$)
    
    \begin{tabularx}{0.85\textwidth}{XX}
        A. (I)、(II)均正确 & B. (I)、(II)均错误\\ C. 仅(I)正确 & D. 仅(II)正确
    \end{tabularx}
\end{frame}

\begin{frame}
    
\end{frame}

\section{填空题}

\begin{frame}
    6.曲线 $y=(x-5)\cdot\sqrt[3]{x^{2}}$ 的下凸区间为:$\underline{\qquad\qquad\qquad\qquad\qquad}$.
\end{frame}

\begin{frame}
    7.设函数 $f(x)=(1+x) \arctan x$,则 $f(x)$ 带皮亚诺余项的三阶麦克劳林展开式为:$\underline{\qquad\qquad\qquad\qquad\qquad}$.
\end{frame}

\begin{frame}
    8.$\displaystyle \int_{0}^{2\pi}\sin^{3}x\cdot e^{\cos x}\mathrm dx = \underline{\qquad\qquad\qquad\qquad\qquad}$.
\end{frame}

\begin{frame}
    9.微分方程
\end{frame}

\begin{frame}
    10.极限 $\displaystyle \lim_{n\rightarrow\infty}\left(\frac{\sqrt{1+\frac{1}{n}}}{n+\frac{1}{n}}+\frac{\sqrt{1+\frac{2}{n}}}{n+\frac{2}{n}}+\cdot\cdot\cdot+\displaystyle\frac{\sqrt{1+\frac{n}{n}}}{n+\frac{n}{n}}\right) = \underline{\qquad\qquad\qquad}$.
\end{frame}

\begin{frame}
    
\end{frame}

\section{计算题}

\begin{frame}
    11.求极限 $\displaystyle \lim_{x\rightarrow0}\frac{\int_{0}^{x}[t-\ln(1+t)]\mathrm dt}{(1+e^{-x})(x-\arctan x)}$.
\end{frame}

\begin{frame}
    12.计算不定积分 $\displaystyle \int\frac{\sqrt{4-x^{2}}}{x}\mathrm dx$.
\end{frame}

\begin{frame}
    13.设函数 $\displaystyle f(x)=\int_{-1}^{1}|x-t|e^{t^{2}}\mathrm dt$,求 $f''(x)$.
\end{frame}

\begin{frame}
    14.设
    \[
    f(x)=\begin{cases}
        \displaystyle\frac{1}{\sqrt{3-2x-x^{2}}}, & 0\le x\le1\\ 
        \displaystyle xe^{-x}, & 1<x\le2
    \end{cases},
    \]
    计算 $\displaystyle \int_{1}^{3}f(x-1)\mathrm dx.$
\end{frame}

\begin{frame}
    
\end{frame}

\begin{frame}
    15.计算反常积分 $\displaystyle \int_{1}^{+\infty}\frac{1}{x^{3}}\arcsin\frac{1}{x} \mathrm dx.$
\end{frame}

\begin{frame}
    16.微分方程 
    
    17.函数性态
\end{frame}

\begin{frame}
    18. 设抛物线 $y=ax^{2}+bx$ 在 $0\le x\le1$ 时 $y\ge0$,且该抛物线与 $x$ 轴及直线 $x=1$ 所围图形的面积为 $\frac{1}{3}$,试确定 $a,b$,使此图形绕 $x$ 轴旋转一周而成的旋转体的体积最小.
\end{frame}

\begin{frame}
    
\end{frame}

\section{证明题}

\begin{frame}
    19.设函数 $f(x)$ 在 $[0,1]$ 上连续,且 $1 < f(x) < 2$,
    证明:
    \[
    1 \le \int_{0}^{1}f(x)\mathrm dx \cdot \int_{0}^{1}\frac{1}{f(x)}\mathrm dx < \frac{9}{8}.
    \]    
\end{frame}

\begin{frame}
    
\end{frame}

\begin{frame}
    
\end{frame}

% -------------------------------------------------------
% \section*{}
% \begin{frame}
% \vspace{25pt}
% \[
% \text{\Huge }
% \]
% \end{frame}

\end{document}