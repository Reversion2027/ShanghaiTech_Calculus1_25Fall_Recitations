\documentclass[]{beamer}
\usepackage[utf8]{inputenc}
\usepackage{xeCJK}
\usepackage{graphicx}
\usepackage{subfigure}
\usepackage{mathtools}
\usepackage{utopia} %font utopia imported
\usetheme{CambridgeUS}
\usecolortheme{dolphin}
\usefonttheme{professionalfonts}
\usepackage{natbib}
\usepackage{hyperref}
\usepackage{fontspec}
\usepackage{setspace}
\usepackage{float}
\usepackage{extarrows}
% \usepackage{enumitem}

\setCJKmainfont{SourceHanSansSC-Regular.otf}[Path=../fonts/, BoldFont=bold.otf]

\setbeamerfont{title}{size=\Large}
\setbeamerfont{subtitle}{size=\small}
\setbeamerfont{date}{size=\small}
\setbeamerfont{institute}{size=\small}

\setstretch{1.3}
% \setlength{\parindent}{2em}
% \setlength{\parskip}{0pt}
% \setlist[itemize]{leftmargin=2em}

% ↓↓↓ Modify this ↓↓↓
\title{高等数学I\quad 习题课16}
\subtitle{级数}
\date[2025.12.30]{2025.12.30}
% ↑↑↑ Modify this ↑↑↑

\author[上海科技大学]{}
\institute[]{上海科技大学}

\begin{document}

\begin{frame}
    \vspace{15pt}
    \titlepage
\end{frame}

\begin{frame}{目录}
    \vfill
    \tableofcontents[hideallsubsections]
    \vfill
\end{frame}

\AtBeginSection[ ]
{
\begin{frame}{目录}
    \vfill
    \tableofcontents[currentsection,hideallsubsections]
    \vfill
\end{frame}
}

\begin{frame}{在开始之前\dots}
    \begin{itemize}
        \item 既然我们已经有了积分这样强有力的分析工具,为什么还需要级数这样一种间断的求和方法?
        \item 电子设备只能将实时信号转化为离散形式,再进行处理与分析
    \end{itemize}    
\end{frame}

\section{数项级数}

\begin{frame}{定义}
    给定数列$\{a_n\}$,称和式
    \[
    \sum_{n=1}^{\infty}a_n=a_1+a_2+\cdots+a_n+\cdots
    \]
    为\textbf{无穷级数},简称\textbf{级数}.和式中的每一项称为级数的项,$a_n$称为级数的通项或一般项.

    有限项和
    \[
    S_n=\sum_{k=1}^{n}a_k=a_1+a_2+\cdots+a_n
    \]
    称为级数的前$n$项部分和.
\end{frame}

\begin{frame}{敛散性}
    若级数$\displaystyle\sum_{n=1}^{\infty}a_n$的部分和数列$\{S_n\}$收敛,且$\displaystyle\lim_{n\rightarrow \infty}S_n=S$,则称级数$\displaystyle\sum_{n=1}^{\infty}a_n$收敛,且收敛到$S$.

    \[
    S=\lim_{n\rightarrow\infty}S_n=\sum_{n=1}^{\infty}a_n
    \]

    若$\{S_n\}$发散则称级数$\displaystyle\sum_{n=1}^{\infty}a_n$发散.
\end{frame}

\begin{frame}{级数的线性运算}
    \begin{align*}
        \sum_{n=1}^{\infty}a_n=S,c\in\mathbb R\quad&\Rightarrow\quad \sum_{n=1}^{\infty}c\ a_n=cS\\
        \sum_{n=1}^{\infty}a_n=S,\ \sum_{n=1}^{\infty}b_n=T \quad&\Rightarrow\quad\sum_{n=1}^{\infty}(a_n+b_n)=S+T
    \end{align*}
\end{frame}

\begin{frame}{敛散性的不变性}
    \begin{itemize}
        \item 增、删、改有限项,级数的敛散性不变
        \item 若级数收敛到$S$,则对其相邻若干项加括号并为一项后,级数仍然收敛到$S\qquad$ (逆命题不成立)
        \item 若级数收敛,则$\displaystyle\lim_{n\rightarrow\infty}a_n=0$
    \end{itemize}
\end{frame}

\begin{frame}[t]{例}
    判断以下级数的敛散性,并求收敛级数的和
    \begin{columns}
        \begin{column}{0.5\textwidth}
            \begin{itemize}
                \item [(1)] $\displaystyle\sum_{n=1}^{\infty}\frac{3^n+(-2)^n}{6^n}$
                \item [(3)] $\displaystyle\sum_{n=1}^{\infty}\ln \frac{n}{n+1}$
            \end{itemize}
        \end{column}
        \begin{column}{0.5\textwidth}
            \begin{itemize}
                \item [(2)] $\displaystyle\sum_{n=1}^{\infty}\frac{1}{n(n+2)}$
                \item [(4)] $\displaystyle\sum_{n=1}^{\infty}\left(\frac{n}{n+1}\right)^n$
            \end{itemize}
        \end{column}
    \end{columns}
\end{frame}

\begin{frame}
    
\end{frame}

\section{正项级数}

\begin{frame}{收敛原理}
    \begin{itemize}
        \item 正项级数:$\forall n,\ a_n\ge 0$的级数
    \end{itemize}
    
    \vspace{5pt}

    \begin{itemize}
        \item 正项级数$\displaystyle\sum_{n=1}^{\infty}a_n$收敛的充要条件是其部分和数列$\{S_n\}$有上界
        \item 推论:若级数$\displaystyle\sum_{n=1}^{\infty}a_n$满足:$\exists\ N>0$,使得当$n>N$时,有$a_n\ge 0$,那么级数$\displaystyle\sum_{n=1}^{\infty}a_n$收敛的充要条件是$\displaystyle\sum_{n=1}^{\infty}a_n$
        的部分和数列$\{S_n\}$有上界.
    \end{itemize}
\end{frame}

\begin{frame}[t]{$p$级数}
    讨论$\displaystyle\sum_{n=1}^{\infty}\frac{1}{n^p}$的敛散性.
\end{frame}

\begin{frame}
    
\end{frame}

\begin{frame}{比较判别法}
    若$\displaystyle\sum_{n=1}^{\infty}a_n$和$\displaystyle\sum_{n=1}^{\infty}b_n$均为正项级数,且$\exists N\in \mathbb Z^+$,使得当$n>N$时,
    \[
    a_n\le b_n
    \]
    那么:
    
    \begin{itemize}
        \item 当$\displaystyle\sum_{n=1}^{\infty}b_n$收敛时,$\displaystyle\sum_{n=1}^{\infty}a_n$也收敛;
        \item 当$\displaystyle\sum_{n=1}^{\infty}a_n$发散时,$\displaystyle\sum_{n=1}^{\infty}b_n$也发散.
    \end{itemize}
\end{frame}

\begin{frame}{比较判别法的极限形式}
    设$a_n\ge 0$且$b_n>0\ (n\in\mathbb N^+)$,又$\displaystyle\lim_{n\rightarrow\infty}\frac{a_n}{b_n}=l$(或$+\infty$),那么
    \begin{itemize}
        \item [(1)] 当$0<l<+\infty$时,级数$\displaystyle\sum_{n=1}^{\infty}a_n$与$\displaystyle\sum_{n=1}^{\infty}b_n$有相同的敛散性;
        \item [(2)] 当$l=0$时,由$\displaystyle\sum_{n=1}^{\infty}b_n$收敛可得$\displaystyle\sum_{n=1}^{\infty}a_n$收敛;
        \item [(3)] 当$l=+\infty$时,由$\displaystyle\sum_{n=1}^{\infty}a_n$发散可得$\displaystyle\sum_{n=1}^{\infty}b_n$发散.
    \end{itemize}
\end{frame}

\begin{frame}{$p$-判别法}
    设$\displaystyle\sum_{n=1}^{\infty}a_n$是正项级数,且$\displaystyle\lim_{n\rightarrow\infty}n^pa_n=l$(或$+\infty$),那么
    \begin{itemize}
        \item [(1)] 当$0\le l<+\infty$,且$p>1$时,$\displaystyle\sum_{n=1}^{\infty}a_n$收敛;
        \item [(2)] 当$0< l\le+\infty$,且$p\le1$时,$\displaystyle\sum_{n=1}^{\infty}a_n$发散.
    \end{itemize}
\end{frame}

\begin{frame}[t]{例11.9, 11.10}
    判断下列级数的敛散性:
    \begin{columns}
        \begin{column}{0.5\textwidth}
            \begin{itemize}
                \item [(1)]$\displaystyle\sum_{n=1}^{\infty}\frac{1}{\sqrt{n^3-n+1}}$
            \end{itemize}
        \end{column}
        \begin{column}{0.5\textwidth}
            \begin{itemize}
                \item [(2)]$\displaystyle\sum_{n=1}^{\infty}a^{\ln\frac1n}\ (a>0)$
            \end{itemize}
        \end{column}
    \end{columns}
\end{frame}

\begin{frame}{比值判别法}
    设$\displaystyle\sum_{n=1}^{\infty}a_n$是正项级数,且$\displaystyle\lim_{n\rightarrow\infty}\frac{a_{n+1}}{a_n}=l$(或$+\infty$),那么
    \begin{itemize}
        \item [(1)] 当$0\le l <1$时,级数$\displaystyle\sum_{n=1}^{\infty}a_n$收敛;
        \item [(2)] 当$1<l<+\infty$时,级数$\displaystyle\sum_{n=1}^{\infty}a_n$发散.
    \end{itemize}
    注意:$l=1$则不可用此方法判断
\end{frame}

\begin{frame}{根值判别法}
    设$\displaystyle\sum_{n=1}^{\infty}a_n$是正项级数,且$\displaystyle\lim_{n\rightarrow\infty}\sqrt[n]{a_n}=l$(或$+\infty$),那么
    \begin{itemize}
        \item [(1)] 当$0\le l <1$时,级数$\displaystyle\sum_{n=1}^{\infty}a_n$收敛;
        \item [(2)] 当$1<l<+\infty$时,级数$\displaystyle\sum_{n=1}^{\infty}a_n$发散.
    \end{itemize}
    注意:$l=1$则不可用此方法判断
\end{frame}

\begin{frame}{积分判别法}
    设$\displaystyle\sum_{n=1}^{\infty}a_n$是正项级数,若非负函数$f(x)$在$[1,+\infty)$上单调减少,且$\forall n\in \mathbb Z^+,\ a_n=f(n)$,
    则级数$\displaystyle\sum_{n=1}^{\infty}a_n$与反常积分$\displaystyle\int_1^{+\infty}f(x)\mathrm dx$的敛散性相同.
\end{frame}

\begin{frame}[t]{例}
    判断下列级数的敛散性:
    \begin{columns}
        \begin{column}{0.5\textwidth}
            \begin{itemize}
                \item [(1)] $\displaystyle\sum_{n=1}^{\infty}\frac{n^3}{3^n}$
                \item [(3)] $\displaystyle\sum_{n=1}^{\infty}\left(2n\arcsin\frac1n\right)^{n/2}$
                \item [(5)] $\displaystyle\sum_{n=1}^{\infty}\frac{(2n-1)!!}{n!}$
            \end{itemize}
        \end{column}
        \begin{column}{0.5\textwidth}
            \begin{itemize}
                \item [(2)] $\displaystyle\sum_{n=1}^{\infty}3^n\cdot\left(\frac{n}{n+1}\right)^{n^2}$
                \item [(4)] $\displaystyle\sum_{n=1}^{\infty}\frac{\arctan n}{n^2+1}$
                \item [(6)] $\displaystyle\sum_{n=1}^{\infty}ne^{-n^2}$
            \end{itemize}
        \end{column}
    \end{columns}
\end{frame}

\begin{frame}
    
\end{frame}

\begin{frame}[t]{习题11 T10}
    证明:
    \begin{itemize}
        \item [(1)] 若$a_n\ge 0$,且$\displaystyle\sum_{n=1}^{\infty}a_n$收敛,则$\displaystyle\sum_{n=1}^{\infty}a_n^2$收敛
    \end{itemize}
\end{frame}

\begin{frame}[t]{习题11 T10}
    证明:
    \begin{itemize}
        \item [(2)] 若$a_n\ge 0$,且数列$\{na_n\}$收敛,则$\displaystyle\sum_{n=1}^{\infty}a_n^2$收敛
    \end{itemize}
\end{frame}

\begin{frame}[t]{习题11 T10}
    证明:
    \begin{itemize}
        \item [(3)] 若$a_n\ge 0,\ b_n\ge 0$,且$\displaystyle\sum_{n=1}^{\infty}a_n$和$\displaystyle\sum_{n=1}^{\infty}b_n$都收敛,则$\displaystyle\sum_{n=1}^{\infty}a_nb_n$和$\displaystyle\sum_{n=1}^{\infty}(a_n+b_n)^2$都收敛;
    \end{itemize}
\end{frame}

\begin{frame}[t]{习题11 T10}
    证明:
    \begin{itemize}
        \item [(4)] 若$a_n\ge 0$,且$\displaystyle\sum_{n=1}^{\infty}a_n$收敛,则$\displaystyle\sum_{n=1}^{\infty}\frac{a_n}{n}$收敛
    \end{itemize}
\end{frame}

\begin{frame}[t]{习题11 T10}
    证明:
    \begin{itemize}
        \item [(5)] 若数列$\{na_n\}$收敛,且$\displaystyle\sum_{n=1}^{\infty}n(a_n-a_{n-1})$收敛,则$\displaystyle\sum_{n=1}^{\infty}a_n$收敛
    \end{itemize}
\end{frame}

\section{交错级数}

\begin{frame}{Leibniz判别法}
    各项正负相间,即形如$\displaystyle\pm\sum_{n=1}^{\infty}(-1)^{n-1}a_n\ (a_n>0)$的级数称为交错级数.

    若交错级数$\displaystyle\sum_{n=1}^{\infty}(-1)^{n-1}a_n$满足:
    \begin{itemize}
        \item [(1)] $0<a_{n+1}\le a_n\ (n\in\mathbb Z^+)$
        \item [(2)] $\lim_{n\rightarrow\infty}a_n=0$
    \end{itemize}
    则级数$\displaystyle\sum_{n=1}^{\infty}(-1)^{n-1}a_n$收敛,且其余项级数满足
    
    $$\left|\sum_{k=n+1}^{\infty}(-1)^{n-1}a_n\right|\le a_{n+1}$$
\end{frame}

\begin{frame}{绝对收敛与条件收敛}
    设$\displaystyle\sum_{n=1}^{\infty}a_n$是任意项级数.若级数$\displaystyle\sum_{n=1}^{\infty}|a_n|$收敛,则称$\displaystyle\sum_{n=1}^{\infty}a_n$绝对收敛;
    若级数$\displaystyle\sum_{n=1}^{\infty}|a_n|$发散而$\displaystyle\sum_{n=1}^{\infty}a_n$收敛,则称级数$\displaystyle\sum_{n=1}^{\infty}a_n$条件收敛.

    \vspace{30pt}

    若$\displaystyle\sum_{n=1}^{\infty}a_n$绝对收敛,则$\displaystyle\sum_{n=1}^{\infty}a_n$收敛.
\end{frame}

\begin{frame}{绝对收敛级数的性质}
    若级数$\displaystyle\sum_{n=1}^{\infty}a_n$绝对收敛,那么任意交换其各项的次序所得的新级数仍绝对收敛,且其和不变.

    \vspace{30pt}

    条件收敛的级数无此性质.
\end{frame}

\begin{frame}[t]{例}
    考察交错级数$\displaystyle\sum_{n=1}^{\infty}(-1)^{n-1}\frac{1}{\sqrt{n}}$的敛散性.
\end{frame}

\begin{frame}[t]{例}
    判断下列级数的敛散性,并说明是绝对收敛还是条件收敛
    \begin{columns}
        \begin{column}{0.5\textwidth}
            \begin{itemize}
                \item [(1)] $\displaystyle\sum_{n=1}^{\infty}(-1)^{n+1}\ln\frac{n}{n+1}$
                \item [(3)] $\displaystyle\sum_{n=1}^{\infty}\left(\frac{1-3n}{3+4n}\right)^{n}$
            \end{itemize}
        \end{column}
        \begin{column}{0.5\textwidth}
            \begin{itemize}
                \item [(2)] $\displaystyle\sum_{n=1}^{\infty}(-1)^{n+1}\frac{(2n+3)!!}{(2n)!!}$
                \item [(4)] $\displaystyle\sum_{n=2}^{\infty}\sin\left(n\pi+\frac{1}{\ln n}\right)$
            \end{itemize}
        \end{column}
    \end{columns}
\end{frame}

\begin{frame}
    
\end{frame}

\end{document}