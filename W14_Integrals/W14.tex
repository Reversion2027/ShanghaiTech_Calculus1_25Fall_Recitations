\documentclass[]{beamer}
\usepackage[utf8]{inputenc}
\usepackage{xeCJK}
\usepackage{graphicx}
\usepackage{subfigure}
\usepackage{mathtools}
\usepackage{utopia} %font utopia imported
\usetheme{CambridgeUS}
\usecolortheme{dolphin}
\usefonttheme{professionalfonts}
\usepackage{natbib}
\usepackage{hyperref}
\usepackage{fontspec}
\usepackage{setspace}
\usepackage{float}
\usepackage{extarrows}
% \usepackage{enumitem}

\setCJKmainfont{SourceHanSansSC-Regular.otf}[Path=../, BoldFont=bold.otf]

\setbeamerfont{title}{size=\Large}
\setbeamerfont{subtitle}{size=\small}
\setbeamerfont{date}{size=\small}
\setbeamerfont{institute}{size=\small}

\setstretch{1.3}
% \setlength{\parindent}{2em}
% \setlength{\parskip}{0pt}
% \setlist[itemize]{leftmargin=2em}

% ↓↓↓ Modify this ↓↓↓
\title{高等数学I\quad 习题课14}
\subtitle{定积分}
\date[2025.12.18]{2025.12.18}
% ↑↑↑ Modify this ↑↑↑

\author[上海科技大学]{}
\institute[]{上海科技大学}

\begin{document}

\begin{frame}
    \vspace{15pt}
    \titlepage
\end{frame}

\begin{frame}{目录}
    \vfill
    \tableofcontents[hideallsubsections]
    \vfill
\end{frame}

\AtBeginSection[ ]
{
\begin{frame}{目录}
    \vfill
    \tableofcontents[currentsection,hideallsubsections]
    \vfill
\end{frame}
}

\section{定积分}

\subsection{几何理解}

\begin{frame}{回到两周前,不定积分}
    \begin{itemize}
        \item 情景:人坐在车内,只能看见速度表$v(t)$,试图找出当前已经行驶过的距离$S(t)$
        \item 做法:每隔固定间隔$\Delta t$,记录速度$v_i$,认为在该时间段内,汽车匀速行驶,距离$S_i=v_i\Delta t$,$\displaystyle S_0=\sum_n v_i\Delta t$.
        \item 随着$\Delta t$不断减小,$S_0$越来越接近真实的距离$S$,即
        \[
        S=\lim_{\mathrm dt\rightarrow 0}\sum_n v_i\mathrm dt=\int_a^b v(t)\mathrm dt
        \]
    \end{itemize}
\end{frame}

\begin{frame}{$\mathrm dt$?}
    $\mathrm dt$代表?
    \begin{enumerate}
        \item 图像中,每个矩形的宽度(显式的$\mathrm dt$项)
        \item 记录速度的时间步长(藏在$v_i$中),决定$n$的大小
    \end{enumerate}
    注意:矩形的高度可以在$[n\Delta T,n\Delta t+\Delta t]$区间上函数的值域内任意选取. 由函数的可积性,可知当区间大小$\Delta t$足够小,最终取值都将收缩到一个点上,因此多个小矩形面积之和也都将拟合为曲线下面积.
\end{frame}

\subsection{定义}

\begin{frame}{课后题 1.(2)}
    使用定义计算定积分
    \[
    \int_0^1 e^x\mathrm dx
    \]    
\end{frame}

\subsection{性质}

\begin{frame}{Schwarz不等式}
    设函数$f(x),g(x)\in C[a,b]$,则
    \[
    \left(\int_a^b f(x)g(x)\mathrm dx\right)^2\le \int_a^bf^2(x)\mathrm dx\int_a^bg^2(x)\mathrm dx
    \]
    可以证明,当条件由$C[a,b]$减弱到$R[a,b]$时,不等式依然成立.
\end{frame}

\subsection{积分中值定理}

\begin{frame}{积分中值定理}
    设函数$f\in C[a,b],g\in R[a,b]$,且$g(x)$在$[a,b]$上不变号,则
    \[
    \exists \xi\in[a,b],\ \text{s.t.}\ \int_a^b f(x)g(x)\mathrm dx=f(\xi)\int_a^bg(x)\mathrm dx
    \]
\end{frame}

\section{微积分基本定理}

\begin{frame}{问题}
    \begin{itemize}
        \item 之前对问题的转换,似乎只是把“根据仪表盘显示求距离”这样一个复杂的问题,转换为了“求$v(t)$曲线下面积”一个同样复杂的问题
        \item 当我们考虑这个问题的更多性质时...
    \end{itemize}
\end{frame}

\subsection{变上限积分}

\begin{frame}{变上限积分?}
    设$v(t)=t(8-t)$. 考虑函数
    \[
    S(T)=\int_0^T v(t)\mathrm dt
    \]
    它的几何意义?它的导数?
\end{frame}

\begin{frame}{变上限积分的导数}
    从两个不同角度:
    \begin{enumerate}
        \item $S(T)$是当前行驶过的距离,其导函数自然应当等于$v(t)$;
        \item 考虑$\mathrm dS$,当$\mathrm dt$足够小,$\mathrm dS=v(t)\mathrm dt$,即$S'(t)=v(t)$
    \end{enumerate}
\end{frame}



\begin{frame}{变上限积分}
    设函数$f\in R[a,b]$,则称函数
    \[
    \Phi(x)=\int_a^x f(t)\mathrm dt,\quad x\in[a,b]
    \]
    为$f$在$[a,b]$上的\textbf{变上限积分函数},简称为变上限积分.
\end{frame}

\begin{frame}{性质}
    \begin{itemize}
        \item $\Phi(x)\in C[a,b]$
        \item 若$f(x)\in C[a,b]$,则$\Phi(x)\in D[a,b]$,且它的导函数\[\Phi'(x)=\frac{\mathrm d}{\mathrm dx}\int_a^xf(t)\mathrm dt=f(x)\]
    \end{itemize}
\end{frame}

\begin{frame}{例5.10}
    求
    \[
    \int_{2x}^0xe^{-t} \mathrm dt
    \]
    的导函数.
\end{frame}

\begin{frame}{例5.11}
    求
    \[
    \lim_{x\rightarrow 0^+}\frac{\displaystyle\int_0^{x^2}\arctan \sqrt{t}\mathrm dt}{\displaystyle\ln(1+x^3)}
    \]
\end{frame}

\subsection{牛顿-莱布尼茨公式}

\begin{frame}{再次回到情景}
    设$v(t)=t(8-t)$. 考虑函数
    \[
    S(T)=\int_0^T v(t)\mathrm dt
    \]
    想要求出自$a$时刻至$b$时刻,汽车行驶的距离
\end{frame}

\begin{frame}{理解}
    \begin{enumerate}
        \item 两处行驶的总距离之差即为所求,$S(b)-S(a)$
        \item $[a,b]$区间上曲线下的面积即为行驶过的距离:$\displaystyle\int_a^b v(t)\mathrm dt$
    \end{enumerate}
    $\Rightarrow$
    $$\displaystyle \int_a^b v(t)\mathrm dt=S(b)-S(a)$$
\end{frame}

\begin{frame}{牛顿-莱布尼茨公式}
    设函数$f\in C[a,b]$,且$F(x)$是$f(x)$在$[a,b]$内的一个原函数,则
    \[
    \int_a^b f(x)\mathrm dx = F(b)-F(a)=:F(x)\Big|_a^b
    \]
\end{frame}

\section{综合练习}

\begin{frame}{例5.9}
   设函数$f\in C[0,1]$,且$f\in D(0,1)$,又$\displaystyle f(1)=2\int_0^{\frac12}xf(x)\mathrm dx$,证明:
   \[
   \exists \xi\in(0,1),\ \text{s.t.}\ f(\xi)+\xi f'(\xi)=0.
   \] 
\end{frame}

\begin{frame}{例5.57 (奇函数、偶函数的对称性)}
    设函数$f(x)\in R[-a,a]$,证明:
    \[
    \int_{-a}^a f(x)\mathrm dx=\left\{
    \begin{array}{ll}
        0,&f\text{为奇函数}\\
        \displaystyle 2\int_0^af(x)\mathrm dx,&f\text{为偶函数}
    \end{array}
    \right.
    \]
\end{frame}

\begin{frame}{例5.58 (周期函数的积分性质)}
    设$f(x)$是$\mathbb R$以$T$为周期的连续函数,证明:
    \[
    \forall a\in \mathbb R,\ \int_a^{a+T}f(x)\mathrm dx=\int_0^Tf(x)\mathrm dx
    \]
\end{frame}

\begin{frame}{例5.60 (三角函数的对称性)}
    计算:
    \[
    I_1=\int_{0}^{\frac{\pi}{2}}\frac{\mathrm dx}{1+\tan x},\qquad I_2=\int_{-\pi}^{\pi}\frac{x\sin x\mathrm dx}{1+\cos^2 x}
    \]
\end{frame}

\begin{frame}{练习}
    求
    \begin{columns}
        \begin{column}{0.5\textwidth}
            \begin{enumerate}
                \item [(1)]$\displaystyle\int_1^2 x\ln x\mathrm dx$
                \item [(2)]$\displaystyle\int_0^1 \arctan x\mathrm dx$
                \item [(3)]$\displaystyle\int_0^{\pi^2/4}\sin\sqrt{x}\mathrm dx$
                \item [(4)]$\displaystyle\int_0^{2\pi}\sin^4x\cos^2x\mathrm dx$
            \end{enumerate}
        \end{column}
        \begin{column}{0.5\textwidth}
            \begin{enumerate}
                \item [(5)] $\displaystyle\int_0^{2\pi}x\cos^2 x\mathrm dx$
                \item [(6)] $\displaystyle\int_0^{\pi/4} \sec^3x\mathrm dx$
                \item [(7)] $\displaystyle\int_0^1 x\sqrt{(1-x^4)^3}\mathrm dx$
                \item [(8)] $\displaystyle\int_{0}^{\pi/4}\frac{x\mathrm dx}{1+\cos 2x}$
            \end{enumerate}
        \end{column}
    \end{columns}
\end{frame}


\end{document}